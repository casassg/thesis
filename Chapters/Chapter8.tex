% Chapter Template

\chapter{Related Work} % Main chapter title

\label{Chapter8} % Change X to a consecutive number; for referencing this chapter elsewhere, use \ref{ChapterX}

There has been a lot of research on big data analytics. Open source tools like Hadoop and Spark have made available scalable, distributed computation to the general public. In addition, thanks to the progress in big data storage, with systems like Cassandra, hosting internal storage has never been easier. However there is not a lot of work regarding using container-orchestration for big data analysis systems. There has been a lot work in both fields, but there is not a lot of work about combining them.

The work presented by \parencite{smack} is probably the closest approach to the system I present here. The main differences are in the actor system and the orchestration platform. Instead of using Mesos for orchestration and Akka for actors, I preferred to use Kubernetes with container microservices instead of actors. The stack described also uses Spark Streaming to deploy a Lambda architecture which could be done in the current infrastructure extending the capabilities of the system. The main reason for avoiding Spark Streaming has been that its micro batching focus could involve losing tweets during the normalization process. Furthermore, the reason to avoid Akka for actors was that it had similar features to Kafka and microservices in Kubernetes. In that way, I considered that adding Akka would make the system more complex and therefore more difficult to maintain.

There has also been previous work trying to approach big data analytics to a higher scalability like this paper from 2014 \parencite{scalableHadoop}. This system proposal is agnostic on how to deploy such systems, leaving the developer to deal with it. As container and container orchestrated systems were not popular at that time, they are not considered. Its main approach to getting these systems to scale is via the use of different Hadoop components. Currently, Spark is preferred over Hadoop, given the increased performance made possible via its memory-based approch to MapReduce.

In \parencite{lambda2014}, we see a similar approach using Hadoop and Pig in the batch layer instead of Cassandra and Spark. They also incorporate a streaming layer for real-time analysis making use of the lambda infrastructure. In \parencite{streamingAnalysis}, we can see Zeppelin used as a dashboard interface similar to our work. However none of this systems made use of the container-orchestrated approach and so lack the scalability and reliablity benefits that they provide.
