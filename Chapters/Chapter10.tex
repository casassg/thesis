% Chapter Template

\chapter{Conclusions} % Main chapter title

\label{Chapter10} % Change X to a consecutive number; for referencing this chapter elsewhere, use \ref{ChapterX}

Container orchestrated technologies make it easier to develop Big Data Analytics systems. Their abstraction layer allows for a separation of responsabilities between developers and system operators, allowing them to work without overlapping their work. This opens a gate for innovation on the Big Data Analytics field, as improvements can be developed separately for infrastructure and software. It also opens a gate to new approaches to system state, extracting the state into the ruling container orchestrated system.

In our case, using container orchestration to recreate Project EPIC infrastructure has proved to be easier to scale. It also, has kept reliance from the previous system, abstracting it and making it part of the orchestration system. It has proved to be an easier to mantain infrastructure, as components are smaller and it's easier to make a more continous development and deployment cycle. In addition, container systems like Docker allows for developers to focus more on the application logic instead of worrying about deployment infrastructure. Finally, container orchestrated systems allows for a better reliance by managing the deployment cycle and ensuring a certain amount of deployed instance at any point.

In conclusion, we have proved that container orchestration systems can become a great option when developing Big Data Analytics infrastructures that require a flexible scaling and high reliability. However there are still some limitations. Will we see this approach become a standard the facto? Or will container orchestrated systems only succeed in transactional infrastructures?

% \section{Software development in microservices architectures}

% Thanks to container-orchestrated systems and the popularization of container systems, software development is easier. With microservices, focus moves out of one component software architecture to organizing the whole system. Now, we can focus on packaging alone components instead of having to packaging all of them at once, which makes the development more flexible. This would allow for a more continuous development in the system allowing for a better performance optimization, and an easier way to keep the system updated with the last set of technologies.
